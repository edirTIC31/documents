\documentclass{beamer}
\setbeamertemplate{navigation symbols}{}
\usepackage[french]{babel}
\usepackage[utf8]{inputenc}

\begin{document}

\date{compilé le \today}
\title{Moyens télécoms}
\author{
	EDIR TIC 31\\
	\small{
        Équipe Départementale d’Intervention Rapide\\
        en Technologies de l’Information et de la Communication\\
        Croix-rouge française - DD de Haute-Garonne
    }
}

\begin{frame}
	\titlepage
\end{frame}

\begin{frame}
    \begin{huge}Maxibox\end{huge}
    \vspace{0.5cm}
    \\
    \begin{itemize}
        \item Bulle WiFi locale
        \item Main courante radio
        \item Téléphonie fixe
        \item Téléphonie mobile DECT (portée de $\sim$100m)
        \item Onduleur \\ (autonomie $>$1h sans électricité)
        \item Format fly-case cubique ($\sim$80cm de coté)
    \end{itemize}
\end{frame}

\begin{frame}
    \begin{huge}Minibox \& Microbox\end{huge}
    \vspace{0.5cm}
    \\
    \begin{itemize}
        \item Bulle WiFi locale
        \item Main courante électronique
        \item Téléphonie fixe
        \item Format malette (respectivement $\sim$8Kg \& $\sim$3Kg)
    \end{itemize}
\end{frame}

\begin{frame}
    \begin{huge}Nanobox\end{huge}
    \vspace{0.5cm}
    \\
    \begin{itemize}
        \item Bulle WiFi locale
        \item Main courante électronique
        \item Format carte de crédit
        \item Utilisable sans connaissance technique
    \end{itemize}
\end{frame}

\begin{frame}
    \begin{huge}Moyen d’accès internet\end{huge}
    \vspace{0.5cm}
    \\
    Les différentes box peuvent être relié à internet :
    \begin{itemize}
        \item Via accès filaire ethernet (câble « RJ45 »)
        \item Via accès WiFi externe
        \item Via accès satellite SES 2E (28.4\degre E)\\
            nécessite une visibilité vers le sud / sud-est :\\
            azimuth 143\degre élévation 33\degre (pour Toulouse)
        \item Pas encore disponible : via 3G
    \end{itemize}
    La connexion d’une box à internet permet de :
    \begin{itemize}
        \item Contacter la box via le réseau téléphonique classique
        \item Accéder à la main courante depuis n’importe où\\
            (par exemple depuis un smartphone en 3G)
    \end{itemize}
\end{frame}

\begin{frame}
    \begin{huge}Interconnection des box\end{huge}
    \vspace{0.5cm}
    \\
    L’interconnection des box peut se faire :
    \begin{itemize}
        \item Via câble ethernet, jusqu’à $\sim$100 m\\
            (l’équipe ne possède pas encore de câble de cette longueur)
        \item Via pont wifi, jusqu’à plusieurs kilomètre\\
            \begin{itemize}
                \item nécessite une vision direct
                \item traverse les vitres ou les bâches des tentes Croix-Rouge
                \item relais possible permettant de créer une ligne brisée
            \end{itemize}
        \item Via internet\\(2 box sont reliés de fait si elles sont reliés à internet)
    \end{itemize}
    L’interconnection des box permet :
    \begin{itemize}
        \item La téléphonie entre les box
        \item L’accès à la main courante d’une autre box
        \item L’accès à internet à travers une autre box
    \end{itemize}
\end{frame}

\end{document}
